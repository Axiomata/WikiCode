\documentclass{article}

\usepackage{amsmath}
\usepackage{amssymb}
\usepackage{cancel}

\usepackage{parskip}
\setlength{\parindent}{0cm}

\begin{document}

\title{ASTS Chapter 1 Exercises}
\author{Andrew Helwer}
\maketitle

\section{Exercise 5}

If $Z_{2}(S) = q > 0$, then the first $q+1$ characters of $S$ must all be the same character. This is the only string that matches itself when shifted to the right by one place. Thus $Z_{3}(S) \ldots Z_{q+2}(S)$ are $q-1 \ldots 0$ respectively.

\section{Exercise 6}

We have the following identities:
\begin{equation}
|B| = r - k
\end{equation}
\begin{equation}
    k' = k - l
\end{equation}
By the definition of $r$, we know $S(r+1) \neq S(r-l+1)$. Given $Z_{k'}(S) > |B|$, assume for the purpose of proof by contradiction that $Z_{k}(S) > |B|$ as well. Then we have:
\begin{align}
    S(k' + |B| + 1) &= S(k + |B| + 1) \\ 
  S(k' + (r-k) + 1) &= S(k + (r-k) + 1) \\
    S((k-l) + (r-k) + 1) &= S(k + (r-k) + 1) \\
    S((\cancel{k}-l) + (r-\cancel{k}) + 1) &= S(\cancel{k} + (r - \cancel{k}) + 1) \\
                                  S(r-l+1) &= S(r+1)
\end{align}
This is a contradiction, given the definition of $r$. Thus $Z_{k}(S) \leq |B|$. Since $Z_{k}(S) \geq |B|$ when $Z_{k'} > |B|$, this means $Z_{k}(S) = |B|$ when $Z_{k'}(S) > |B|$.

\section{Exercise 7}

The optimization outlined in exercise 6 will result in negligible speedup due to the addition of an extra case.

\end{document}
